% EJC papers *must* begin with the following two lines. 
\documentclass[12pt]{article}
\usepackage{e-jc}

% Please remove all other commands that change parameters such as
% margins or pagesizes.

% only use standard LaTeX packages
% only include packages that you actually need

% we recommend these ams packages
\usepackage{amsthm,amsmath,amssymb}

% we recommend the graphicx package for importing figures
\usepackage{graphicx}

% use this command to create hyperlinks (optional and recommended)
\usepackage[colorlinks=true,citecolor=black,linkcolor=black,urlcolor=blue]{hyperref}

% use these commands for typesetting doi and arXiv references in the bibliography
\newcommand{\doi}[1]{\href{http://dx.doi.org/#1}{\texttt{doi:#1}}}
\newcommand{\arxiv}[1]{\href{http://arxiv.org/abs/#1}{\texttt{arXiv:#1}}}

% all overfull boxes must be fixed; 
% i.e. there must be no text protruding into the margins


% declare theorem-like environments
\theoremstyle{plain}
\newtheorem{theorem}{Theorem}
\newtheorem{lemma}[theorem]{Lemma}
\newtheorem{corollary}[theorem]{Corollary}
\newtheorem{proposition}[theorem]{Proposition}
\newtheorem{fact}[theorem]{Fact}
\newtheorem{observation}[theorem]{Observation}
\newtheorem{claim}[theorem]{Claim}

\theoremstyle{definition}
\newtheorem{definition}[theorem]{Definition}
\newtheorem{example}[theorem]{Example}
\newtheorem{conjecture}[theorem]{Conjecture}
\newtheorem{open}[theorem]{Open Problem}
\newtheorem{problem}[theorem]{Problem}
\newtheorem{question}[theorem]{Question}

\theoremstyle{remark}
\newtheorem{remark}[theorem]{Remark}
\newtheorem{note}[theorem]{Note}

%%%%%%%%%%%%%%%%%%%%%%%%%%%%%%%%%%%%%%%%%%%%%%%%%%%%%%%

% if needed include a line break (\\) at an appropriate place in the title

\title{\bf An Extension of a Result of Das-Gan-Sudakov}

% input author, affilliation, address and support information as follows;
% the address should include the country, and does not have to include
% the street address 

\author{Erik Waingarten\\
\small Department of Mathematics\\[-0.8ex]
\small Massachusetts Institute of Technology\\[-0.8ex] 
\small Cambridge, MA\\
\small\tt eaw@mit.edu\\
\and
Fermi Ma\\
\small Department of Mathematics\\[-0.8ex]
\small Massachusetts Institute of Technology\\[-0.8ex]
\small Cambridge, MA\\
\small\tt fermima@mit.edu
\and
Peter Kleinhenz\\
\small Department of Mathematics\\[-0.8ex]
\small Massachusetts Institute of Technology\\[-0.8ex]
\small Cambridge, MA\\
\small\tt pkleinhe@mit.edu
}

% \date{\dateline{submission date}{acceptance date}\\
% \small Mathematics Subject Classifications: comma separated list of
% MSC codes available from http://www.ams.org/mathscinet/freeTools.html}

\date{\dateline{August 1, 2014}{XX}\\
\small Mathematics Subject Classifications: XX}

\begin{document}

\maketitle

% E-JC papers must include an abstract. The abstract should consist of a
% succinct statement of background followed by a listing of the
% principal new results that are to be found in the paper. The abstract
% should be informative, clear, and as complete as possible. Phrases
% like "we investigate..." or "we study..." should be kept to a minimum
% in favor of "we prove that..."  or "we show that...".  Do not
% include equation numbers, unexpanded citations (such as "[23]"), or
% any other references to things in the paper that are not defined in
% the abstract. The abstract will be distributed without the rest of the
% paper so it must be entirely self-contained.

\begin{abstract}
  A result of Shagnik Das, Wenying Gan, and Benny Sudakov shows that

  % keywords are optional
  \bigskip\noindent \textbf{Keywords:} graph reconstruction
  conjecture; Broglington manifold; Pipletti's classification
\end{abstract}

\section{Introduction}

Sperner's Theorem is one of the central results in extremal set theory. 

Denote by $B_n$ the poset of subsets of $[n]$, where $F_1 \leq F_2$ if $F_1 \subseteq F_2$ as sets. A 2-chain is any pair of sets $F_1$ and $F_2$ such that $F_1 \subset F_2$. Sperner's Theorem states that any family of $B_n$ with no 2-chains has at most $\binom{n}{\lfloor n/2 \rfloor}$, a bound that can be shown to be tight by considering all the subsets of $[n]$ with size $\lfloor n/2 \rfloor$. Kleitman considered the related question of determining the minimum number of a 2-chains that must appear in a family of sets larger than $\binom{n}{\lfloor n/2 \rfloor}$. Das, Gan, and Sudakov considered this problem independently, and completely characterized all extremal families.

%%%%%%%%%%%%%%%%%%%%%%%%%%%%%%%%%%%%%%%%%%%%%%%%%%%%%%%
\section{Proof of Theorem~\ref{Thm:FabGraphs}}

In this section we complete the proof of Theorem~\ref{Thm:FabGraphs}.

\begin{proof}[Proof of Theorem~\ref{Thm:FabGraphs}]
Let $G$ be a graph... Hence
 % use the amsmath align environment for multi-line equations
  \begin{align}
    |X| &= abcdefghijklmnopqrstuvwxyz \nonumber\\
    &= \alpha\beta\gamma
  \end{align}
  This completes the proof of Theorem~\ref{Thm:FabGraphs}.
\end{proof}

\begin{figure}[!h]
  \begin{center}
    % use \includegraphics to import figures 
    % \includegraphics{filename}
  \end{center}
  \caption{\label{fig:InformativeFigure} Here is an informative
    figure.}
\end{figure}

%%%%%%%%%%%%%%%%%%%%%%%%%%%%%%%%%%%%%%%%%%%%%%%%%%%%%%%
\subsection*{Acknowledgements}
Thanks to Professor Querty for suggesting the proof of
Lemma~\ref{lem:Technical}.

%%%%%%%%%%%%%%%%%%%%%%%%%%%%%%%%%%%%%%%%%%%%%%%%%%%%%%%
% \bibliographystyle{plain} 
% \bibliography{myBibFile} 
% If you use BibTeX to create a bibliography
% then copy and past the contents of your .bbl file into your .tex file

\begin{thebibliography}{10}

\bibitem{Bollobas} B{\'e}la Bollob{\'a}s.  \newblock Almost every
  graph has reconstruction number three.  \newblock {\em J. Graph
    Theory}, 14(1):1--4, 1990.

\bibitem{WikipediaReconstruction} Wikipedia contributors.  \newblock
  Reconstruction conjecture.  \newblock {\em Wikipedia, the free
    encyclopedia}, 2011.

\bibitem{FGH} J.~Fisher, R.~L. Graham, and F.~Harary.  \newblock A
  simpler counterexample to the reconstruction conjecture for
  denumerable graphs.  \newblock {\em J. Combinatorial Theory Ser. B},
  12:203--204, 1972.

\bibitem{HHRT} Edith Hemaspaandra, Lane~A. Hemaspaandra,
  Stanis{\l}aw~P. Radziszowski, and Rahul Tripathi.  \newblock
  Complexity results in graph reconstruction.  \newblock {\em Discrete
    Appl. Math.}, 155(2):103--118, 2007.

\bibitem{Kelly} Paul~J. Kelly.  \newblock A congruence theorem for
  trees.  \newblock {\em Pacific J. Math.}, 7:961--968, 1957.

\bibitem{KSU} Masashi Kiyomi, Toshiki Saitoh, and Ryuhei Uehara.
  \newblock Reconstruction of interval graphs.  \newblock In {\em
    Computing and combinatorics}, volume 5609 of {\em Lecture Notes in
    Comput. Sci.}, pages 106--115. Springer, 2009.

\bibitem{RM} S.~Ramachandran and S.~Monikandan.  \newblock Graph
  reconstruction conjecture: reductions using complement, connectivity
  and distance.  \newblock {\em Bull. Inst. Combin. Appl.},
  56:103--108, 2009.

\bibitem{RR} David Rivshin and Stanis{\l}aw~P. Radziszowski.
  \newblock The vertex and edge graph reconstruction numbers of small
  graphs.  \newblock {\em Australas. J. Combin.}, 45:175--188, 2009.

\bibitem{Stockmeyer} Paul~K. Stockmeyer.  \newblock The falsity of the
  reconstruction conjecture for tournaments.  \newblock {\em J. Graph
    Theory}, 1(1):19--25, 1977.

\bibitem{Ulam} S.~M. Ulam.  \newblock {\em A collection of
    mathematical problems}.  \newblock Interscience Tracts in Pure and
  Applied Mathematics, no. 8.  Interscience Publishers, New
  York-London, 1960.

\end{thebibliography}

\end{document}
