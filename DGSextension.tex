% EJC papers *must* begin with the following two lines. 
\documentclass[12pt]{article}
\usepackage{e-jc}

% Please remove all other commands that change parameters such as
% margins or pagesizes.

% only use standard LaTeX packages
% only include packages that you actually need

% we recommend these ams packages
\usepackage{amsthm,amsmath,amssymb}
\usepackage{verbatim}

% we recommend the graphicx package for importing figures
\usepackage{graphicx}

% use this command to create hyperlinks (optional and recommended)
\usepackage[colorlinks=true,citecolor=black,linkcolor=black,urlcolor=blue]{hyperref}

% use these commands for typesetting doi and arXiv references in the bibliography
\newcommand{\doi}[1]{\href{http://dx.doi.org/#1}{\texttt{doi:#1}}}
\newcommand{\arxiv}[1]{\href{http://arxiv.org/abs/#1}{\texttt{arXiv:#1}}}

% all overfull boxes must be fixed; 
% i.e. there must be no text protruding into the margins


% declare theorem-like environments
\theoremstyle{plain}
\newtheorem{theorem}{Theorem}
\newtheorem{lemma}[theorem]{Lemma}
\newtheorem{corollary}[theorem]{Corollary}
\newtheorem{proposition}[theorem]{Proposition}
\newtheorem{fact}[theorem]{Fact}
\newtheorem{observation}[theorem]{Observation}
\newtheorem{claim}[theorem]{Claim}

\theoremstyle{definition}
\newtheorem{definition}[theorem]{Definition}
\newtheorem{example}[theorem]{Example}
\newtheorem{conjecture}[theorem]{Conjecture}
\newtheorem{open}[theorem]{Open Problem}
\newtheorem{problem}[theorem]{Problem}
\newtheorem{question}[theorem]{Question}

\theoremstyle{remark}
\newtheorem{remark}[theorem]{Remark}
\newtheorem{note}[theorem]{Note}
\newcommand{\F}{\mathcal{F}}
\newcommand{\A}{\mathcal{A}}
\newcommand{\B}{\mathcal{B}}

%%%%%%%%%%%%%%%%%%%%%%%%%%%%%%%%%%%%%%%%%%%%%%%%%%%%%%%

% if needed include a line break (\\) at an appropriate place in the title

\title{\bf An Extension of a Result of Das-Gan-Sudakov}

% input author, affilliation, address and support information as follows;
% the address should include the country, and does not have to include
% the street address 

\author{Erik Waingarten\\
\small Department of Mathematics\\[-0.8ex]
\small Massachusetts Institute of Technology\\[-0.8ex] 
\small Cambridge, MA\\
\small\tt eaw@mit.edu\\
\and
Fermi Ma\\
\small Department of Mathematics\\[-0.8ex]
\small Massachusetts Institute of Technology\\[-0.8ex]
\small Cambridge, MA\\
\small\tt fermima@mit.edu
\and
Peter Kleinhenz\\
\small Department of Mathematics\\[-0.8ex]
\small Massachusetts Institute of Technology\\[-0.8ex]
\small Cambridge, MA\\
\small\tt pkleinhe@mit.edu
}

% \date{\dateline{submission date}{acceptance date}\\
% \small Mathematics Subject Classifications: comma separated list of
% MSC codes available from http://www.ams.org/mathscinet/freeTools.html}

\date{\dateline{August 1, 2014}{XX}\\
\small Mathematics Subject Classifications: XX}

\begin{document}

\maketitle

% E-JC papers must include an abstract. The abstract should consist of a
% succinct statement of background followed by a listing of the
% principal new results that are to be found in the paper. The abstract
% should be informative, clear, and as complete as possible. Phrases
% like "we investigate..." or "we study..." should be kept to a minimum
% in favor of "we prove that..."  or "we show that...".  Do not
% include equation numbers, unexpanded citations (such as "[23]"), or
% any other references to things in the paper that are not defined in
% the abstract. The abstract will be distributed without the rest of the
% paper so it must be entirely self-contained.

\begin{abstract}
  A result of Shagnik Das, Wenying Gan, and Benny Sudakov shows that

  % keywords are optional
  \bigskip\noindent \textbf{Keywords:} graph reconstruction
  conjecture; Broglington manifold; Pipletti's classification
\end{abstract}

\section{Introduction}

Let $B_n$ denote the poset of all subsets of $[n]$, where $F_1 \leq F_2$ if $F_1 \subseteq F_2$ as sets. A 2-chain is any pair of sets $F_1$ and $F_2$ such that $F_1 \subset F_2$. Sperner's Theorem states that any family of $B_n$ with no 2-chains has at most $\binom{n}{\lfloor n/2 \rfloor}$, a bound that can be shown to be tight by considering all the subsets of $[n]$ with size $\lfloor n/2 \rfloor$. Kleitman considered the related question of determining the minimum number of a 2-chains that must appear in a family of sets larger than $\binom{n}{\lfloor n/2 \rfloor}$. Das, Gan, and Sudakov considered this problem independently, and completely characterized all extremal families.

We analyze this problem for posets of the form $B_n / G$, where $G$ is subgroup of $S_n$.

%%%%%%%%%%%%%%%%%%%%%%%%%%%%%%%%%%%%%%%%%%%%%%%%%%%%%%%
\section{When $G$ is Generated by a Transposition}
\begin{theorem} For any $s > M$, there exists a family $\F$ of $s$ elements of $B_n/G$ that minimizes the number of 2-chains such that if $A \in \F$ is of maximal cardinality, where we write $|A| = \dfrac{n}{2} + m$, then for any $B \subset A$ (in this case, $\subset$ corresponds to the $B_n/G$ definition of $<$) with $|B| \geq \dfrac{n}{2} - m + 1$, $B \in \F$.
\end{theorem}

What we do is the following to show that this holds for $G$ generated by the transposition $(12)$:

1. Lets take all the elements in the $\frac{n}{2} + m$ level that have something in the $l$th level which is not included in the family. Lets also assume that there is an injective map from one to the other, where $M(A) \subset A$. 

2. Lets assume also that $l > 1$. 

3. So we show that the $A$ shadows are less than the $M(A)$ greater elements. This means that if we move the elements from $A$ to elements in $M(A)$, we would be decreasing the number of 2-chains. 

\begin{proof}
Suppose $|A| = \frac{n}{2} + m$, and $|M(A)| = \frac{n}{2} + m - l$.

Then we want to compute
\[L = \sum_{A \in \A} \sum_{i=1}^{l-1} |\partial^iA| \]
and 
\[G = \sum_{A \in \A} \sum_{i=1}^{l-1} |U^i(M(A)) \cap \F| \]
Here $L$ stands to for the number of 2-chains lost, and $G$ stands for the number of 2-chains gained. We want to say that $G \leq L$. 

We first compute $L$ by spliting up $A$ in 2 cases:\\
1. $\{ 1, 2 \} \not\subset A$. In this case, whatever we chose to remove will get us a subset of $A$ that is below it. So the number of 2-chains lost involving this kind of $A$ is
\[ \sum_{i=1}^{l-1} \dbinom{\frac{n}{2} + m}{i} \]
2. $\{ 1, 2 \} \subset A$. In this case, it doesn't matter whether we remove $1$ or $2$, so we count the number of sets in the following manner. First, we assume that we remove either 1 or 2 or none. In the other case, we remove 1 and 2.
\[ \sum_{i=1}^{l-1} \left(\dbinom{\frac{n}{2} + m - 1}{i} + \dbinom{\frac{n}{2}+m-2}{i-2} \right) \]
So we have computed $L$ to be the sum over all $A \in \A$:
\[ L = \sum_{\{1, 2\} \not\subset A \in \A} \sum_{i=1}^{l-1} \dbinom{\frac{n}{2} + m}{i} + \sum_{\{1, 2\} \subset A \in \A} \sum_{i=1}^{l-1} \left( \dbinom{\frac{n}{2}+m-1}{i} + \dbinom{\frac{n}{2}+m-2}{i-2}\right) \]

Likewise, we can do the same for $G$, except now we are adding elements from the complements of $M(A)$. We can also assume that everything that we add is in $\F$ in the worst case
\[ G \leq \sum_{1 \in M(A) \text{ or } 2 \in M(A)} \sum_{i=1}^{l-1} \dbinom{\frac{n}{2}-m+l}{i} + \sum_{1, 2 \notin M(A)} \sum_{i=1}^{l-1} \left( \dbinom{\frac{n}{2}-m+l-1}{i} + \dbinom{\frac{n}{2}-m+l-2}{i-2} \right) \]

We know that $\dbinom{l}{k} > \dbinom{l-1}{k} + \dbinom{l-2}{k-2}$, so in order to show that 
\[ G \leq L \]
it sufficies to show that
\[ \dbinom{\frac{n}{2}-m+l}{i} \leq \dbinom{\frac{n}{2}+m-1}{i} + \dbinom{\frac{n}{2}+m-2}{i-2} \]
Which is true, since $l \leq 2m-1$.
\end{proof}

4. Now we allow $l = 1$. This is proved in the paper easily. 

5. Now we assume that there is no matching.

We use the Lemma given in the paper: Let $G$ be a bipartite graph on $U \cup V$ with minimum degree $\delta_U \geq 1$ in $U$ and maximum degree $\Delta_V$ in $V$. Suppose there is no matching from $U$ to $V$. Then there exist nonempty subsets $U_1 \subset U$ and $V_1 \subset V$ with a perfect matching $M: U_1 \to V_1$ and $e(U_1,V) + E(U \setminus U_1,V_1) \leq |U_1|\Delta_V$.

We claim that the auxiliary graph satisfies the conditions of the lemma with $U = \A, V = \B, \Delta_V = |U^l(M(A)) \cap F|$.

We define $\A_1$ and $\B_1$ and $\tilde{\F}$ as in the paper, and we get that the number of chains we are losing is $|\A_1||\delta^l A| - e(\A_1,\B)$ (where $A$ can be any set of size $n/2 + m$). Same argument from above shows that $|\delta^l A| > |U^l(M(A)) \cap F|$, so the same argument from the paper can be copied. In particular, we conclude that $\tilde{\F}$ has fewer 2-chains than $\F$, contradicting the optimality of $\F$.


%%%%%%%%%%%%%%%%%%%%%%%%%%%%%%%%%%%%%%%%%%%%%%%%%%%%%%%
\section{A Counterexample for a Different $G$}

This result does not extend to $B_n / G$ for any $G$. In particular, let $n = 5$ and let $G$ be the group generated by the transpositions $(12), (23),$ and $(34)$.

%%%%%%%%%%%%%%%%%%%%%%%%%%%%%%%%%%%%%%%%%%%%%%%%%%%%%%%
\subsection*{Acknowledgements}
We thank Jonathan Novak (MIT) for guiding our research and providing helpful discussions.

%%%%%%%%%%%%%%%%%%%%%%%%%%%%%%%%%%%%%%%%%%%%%%%%%%%%%%%
% \bibliographystyle{plain} 
% \bibliography{myBibFile} 
% If you use BibTeX to create a bibliography
% then copy and past the contents of your .bbl file into your .tex file

\begin{thebibliography}{10}


\end{thebibliography}

\end{document}
