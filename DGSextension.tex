% EJC papers *must* begin with the following two lines. 
\documentclass[12pt]{article}
\usepackage{e-jc}

% Please remove all other commands that change parameters such as
% margins or pagesizes.

% only use standard LaTeX packages
% only include packages that you actually need

% we recommend these ams packages
\usepackage{amsthm,amsmath,amssymb}
\usepackage{verbatim}

% we recommend the graphicx package for importing figures
\usepackage{graphicx}

% use this command to create hyperlinks (optional and recommended)
\usepackage[colorlinks=true,citecolor=black,linkcolor=black,urlcolor=blue]{hyperref}

% use these commands for typesetting doi and arXiv references in the bibliography
\newcommand{\doi}[1]{\href{http://dx.doi.org/#1}{\texttt{doi:#1}}}
\newcommand{\arxiv}[1]{\href{http://arxiv.org/abs/#1}{\texttt{arXiv:#1}}}

% all overfull boxes must be fixed; 
% i.e. there must be no text protruding into the margins


% declare theorem-like environments
\theoremstyle{plain}
\newtheorem{theorem}{Theorem}
\newtheorem{lemma}[theorem]{Lemma}
\newtheorem{corollary}[theorem]{Corollary}
\newtheorem{proposition}[theorem]{Proposition}
\newtheorem{fact}[theorem]{Fact}
\newtheorem{observation}[theorem]{Observation}
\newtheorem{claim}[theorem]{Claim}

\theoremstyle{definition}
\newtheorem{definition}[theorem]{Definition}
\newtheorem{example}[theorem]{Example}
\newtheorem{conjecture}[theorem]{Conjecture}
\newtheorem{open}[theorem]{Open Problem}
\newtheorem{problem}[theorem]{Problem}
\newtheorem{question}[theorem]{Question}

\theoremstyle{remark}
\newtheorem{remark}[theorem]{Remark}
\newtheorem{note}[theorem]{Note}
\newcommand{\F}{\mathcal{F}}
\newcommand{\A}{\mathcal{A}}
\newcommand{\B}{\mathcal{B}}

%%%%%%%%%%%%%%%%%%%%%%%%%%%%%%%%%%%%%%%%%%%%%%%%%%%%%%%

% if needed include a line break (\\) at an appropriate place in the title

\title{\bf An Extension of a Result of Das-Gan-Sudakov}

% input author, affilliation, address and support information as follows;
% the address should include the country, and does not have to include
% the street address 

\author{Erik Waingarten\\
\small Department of Mathematics\\[-0.8ex]
\small Massachusetts Institute of Technology\\[-0.8ex] 
\small Cambridge, MA\\
\small\tt eaw@mit.edu\\
\and
Fermi Ma\\
\small Department of Mathematics\\[-0.8ex]
\small Massachusetts Institute of Technology\\[-0.8ex]
\small Cambridge, MA\\
\small\tt fermima@mit.edu
}

% \date{\dateline{submission date}{acceptance date}\\
% \small Mathematics Subject Classifications: comma separated list of
% MSC codes available from http://www.ams.org/mathscinet/freeTools.html}

\date{\dateline{August 1, 2014}{XX}\\
\small Mathematics Subject Classifications: XX}

\begin{document}

\maketitle

% E-JC papers must include an abstract. The abstract should consist of a
% succinct statement of background followed by a listing of the
% principal new results that are to be found in the paper. The abstract
% should be informative, clear, and as complete as possible. Phrases
% like "we investigate..." or "we study..." should be kept to a minimum
% in favor of "we prove that..."  or "we show that...".  Do not
% include equation numbers, unexpanded citations (such as "[23]"), or
% any other references to things in the paper that are not defined in
% the abstract. The abstract will be distributed without the rest of the
% paper so it must be entirely self-contained.

\begin{abstract}
  A result of Shagnik Das, Wenying Gan, and Benny Sudakov shows that

  % keywords are optional
  \bigskip\noindent \textbf{Keywords:} graph reconstruction
  conjecture; Broglington manifold; Pipletti's classification
\end{abstract}

\section{Introduction}

Let $B_n$ denote the poset of all subsets of $[n]$, where $F_1 \leq F_2$ if $F_1 \subseteq F_2$ as sets. A 2-chain is any pair of sets $F_1$ and $F_2$ such that $F_1 \subset F_2$. Sperner's Theorem states that any family of $B_n$ with no 2-chains has at most $\binom{n}{\lfloor n/2 \rfloor}$, a bound that can be shown to be tight by considering all the subsets of $[n]$ with size $\lfloor n/2 \rfloor$. Kleitman considered the related question of determining the minimum number of a 2-chains that must appear in a family of sets larger than $\binom{n}{\lfloor n/2 \rfloor}$. Das, Gan, and Sudakov considered this problem independently, and completely characterized all extremal families.

We analyze this problem for posets of the form $B_n / G$, where $G$ is subgroup of $S_n$.

%%%%%%%%%%%%%%%%%%%%%%%%%%%%%%%%%%%%%%%%%%%%%%%%%%%%%%%
\section{Groups Generated By a Transposition}

In this section, we restrict our attention to posets of the form $B_n / G$, where $G$ is a subgroup of the symmetric group $S_n$ on $n$ elements generated by a single transposition. WLOG, let the transposition be $(12)$, and denote by $G_{(12)}$ the two-element group generated by that transposition

Let the $i$th level of $B_n$ consist of all subsets $F$ of size $i$. Then Sperner's Theorem states that no antichain in $B_n$ is larger than the largest level $P_i$ [Stanley]. [Stanley] shows how to extend the theorem to $B_n / G$ for all groups $G$. 

The $i$th level of $B_n / G_{(12)}$ has size $\binom{n-2}{i-2} + \binom{n-1}{i}$, where the first term in the sum counts the number of sets that include both 1 and 2, while the second term counts the sets where at most one element of $\{1,2\}$ is included. Because $B_n / G$ is rank-symmetric, the largest level has $\binom{n-2}{\lfloor n/2 \rfloor -2} + \binom{n-1}{\lfloor n/2 \rfloor}$ elements, which is also the size of the largest family containing no 2-chains [Stanley]

%must add citations

Theorem 1.2 of [Das] completely characterizes the minimum number of 2-chains in families $\F$ of subsets of $B_n$. We present a similar result for families $\F$ of subsets of $B_n / G$, where $G$ is a group generated by a transposition.

Let $m = \binom{n-2}{\lfloor n/2 \rfloor -2} + \binom{n-1}{\lfloor n/2 \rfloor}$.

\begin{theorem} For any $s \geq m$, let $r \in \frac{1}{2}\mathbb{N}$ be the unique half-integer such that $\sum_{i = \frac{n}{2} - r +1}^{\frac{n}{2} + r -1} |P_i| < s \leq \sum_{i = \frac{n}{2}-r}^{\frac{n}{2}+r}|P_i|$. There exists a family of subsets $\F$ of $B_n / G_{(12)}$ where $|\F| = s$ that minimizes the number of 2-chains, and satisfies the following properties:

1) For every $A \in \F$, $n/2 - r \leq |A| \leq n/2 + r$

2) For any $A \in B_n / G_{(12)}$ with $n/2 - r + 1 \leq |A| \leq n/2 + r - 1$, we have $A \in \F$.
\end{theorem}

\begin{proof}
\end{proof}

%%%%%%%%%%%%%%%%%%%%%%%%%%%%%%%%%%%%%%%%%%%%%%%%%%%%%%%
\section{A Counterexample for a Different $G$}

This result does not extend to $B_n / G$ for any $G$. In particular, let $n = 6$ and let $G$ be the group generated by the transpositions $(12), (23),$ and $(34)$. An the following figure gives an example of a collection of 12 sets that is not taken by the middle levels and has less 2-chains than the collection of the 12 middle sets.

\includegraphic{counterexamplegraph.pdf}

%%%%%%%%%%%%%%%%%%%%%%%%%%%%%%%%%%%%%%%%%%%%%%%%%%%%%%%
\subsection*{Acknowledgements}
We thank Jonathan Novak (MIT) for guiding our research and providing helpful discussions.

%%%%%%%%%%%%%%%%%%%%%%%%%%%%%%%%%%%%%%%%%%%%%%%%%%%%%%%
% \bibliographystyle{plain} 
% \bibliography{myBibFile} 
% If you use BibTeX to create a bibliography
% then copy and past the contents of your .bbl file into your .tex file

\begin{thebibliography}{10}

\end{thebibliography}

\end{document}
