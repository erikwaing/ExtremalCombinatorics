\documentclass[11pt]{amsart}
\usepackage{graphicx}    % needed for including graphics e.g. EPS, PS
\topmargin -1.5cm        % read Lamport p.163
\oddsidemargin -0.04cm   % read Lamport p.163
\evensidemargin -0.04cm  % same as oddsidemargin but for left-hand pages
\textwidth 16.59cm
\textheight 21.94cm 
%\pagestyle{empty}       % Uncomment if don't want page numbers
\parskip 7.2pt           % sets spacing between paragraphs
%\renewcommand{\baselinestretch}{1.5} % Uncomment for 1.5 spacing between lines
\usepackage{amsmath}
\usepackage{amsfonts}
\usepackage{verbatim}

\newtheorem{theorem}{Theorem}[section]
\newtheorem{lemma}[theorem]{Lemma}

\theoremstyle{definition}
\newtheorem{definition}[theorem]{Definition}
\newtheorem{example}[theorem]{Example}
\newtheorem{xca}[theorem]{Exercise}
\newtheorem{cor}[theorem]{Corollary}

\theoremstyle{remark}
\newtheorem{remark}[theorem]{Remark}

\parindent 0pt		 % sets leading space for paragraphs
\newcommand{\F}{\mathcal{F}}
\newcommand{\A}{\mathcal{A}}
\newcommand{\B}{\mathcal{B}}
\author{Erik Waingarten \and Fermi Ma}
\title{Extending Proposition 2.1 to $B_n/G$}


\begin{document}         
\maketitle

\textbf{Proposition 2.1} (extended version): For any $s > M$, there exists a family $\F$ of $s$ elements of $B_n/G$ that minimizes the number of 2-chains such that if $A \in \F$ is of maximal cardinality, where we write $|A| = \dfrac{n}{2} + m$, then for any $B \subset A$ (in this case, $\subset$ corresponds to the $B_n/G$ definition of $<$) with $|B| \geq \dfrac{n}{2} - m + 1$, $B \in \F$.

What we do is the following to show that this holds for $G$ generated by the transposition $(12)$:

1. Lets take all the elements in the $\frac{n}{2} + m$ level that have something in the $l$th level which is not included in the family. Lets also assume that there is an injective map from one to the other, where $M(A) \subset A$. 

2. Lets assume also that $l > 1$. 

3. So we show that the $A$ shadows are less than the $M(A)$ greater elements. This means that if we move the elements from $A$ to elements in $M(A)$, we would be decreasing the number of 2-chains. 

\begin{proof}
Suppose $|A| = \frac{n}{2} + m$, and $|M(A)| = \frac{n}{2} + m - l$.

Then we want to compute
\[L = \sum_{A \in \A} \sum_{i=1}^{l-1} |\partial^iA| \]
and 
\[G = \sum_{A \in \A} \sum_{i=1}^{l-1} |U^i(M(A)) \cap \F| \]
Here $L$ stands to for the number of 2-chains lost, and $G$ stands for the number of 2-chains gained. We want to say that $G \leq L$. 

We first compute $L$ by spliting up $A$ in 2 cases:\\
1. $\{ 1, 2 \} \not\subset A$. In this case, whatever we chose to remove will get us a subset of $A$ that is below it. So the number of 2-chains lost involving this kind of $A$ is
\[ \sum_{i=1}^{l-1} \dbinom{\frac{n}{2} + m}{i} \]
2. $\{ 1, 2 \} \subset A$. In this case, it doesn't matter whether we remove $1$ or $2$, so we count the number of sets in the following manner. First, we assume that we remove either 1 or 2 or none. In the other case, we remove 1 and 2.
\[ \sum_{i=1}^{l-1} \left(\dbinom{\frac{n}{2} + m - 1}{i} + \dbinom{\frac{n}{2}+m-2}{i-2} \right) \]
So we have computed $L$ to be the sum over all $A \in \A$:
\[ L = \sum_{\{1, 2\} \not\subset A \in \A} \sum_{i=1}^{l-1} \dbinom{\frac{n}{2} + m}{i} + \sum_{\{1, 2\} \subset A \in \A} \sum_{i=1}^{l-1} \left( \dbinom{\frac{n}{2}+m-1}{i} + \dbinom{\frac{n}{2}+m-2}{i-2}\right) \]

Likewise, we can do the same for $G$, except now we are adding elements from the complements of $M(A)$. We can also assume that everything that we add is in $\F$ in the worst case
\[ G \leq \sum_{1 \in M(A) \text{ or } 2 \in M(A)} \sum_{i=1}^{l-1} \dbinom{\frac{n}{2}-m+l}{i} + \sum_{1, 2 \notin M(A)} \sum_{i=1}^{l-1} \left( \dbinom{\frac{n}{2}-m+l-1}{i} + \dbinom{\frac{n}{2}-m+l-2}{i-2} \right) \]

We know that $\dbinom{l}{k} > \dbinom{l-1}{k} + \dbinom{l-2}{k-2}$, so in order to show that 
\[ G \leq L \]
it sufficies to show that
\[ \dbinom{\frac{n}{2}-m+l}{i} \leq \dbinom{\frac{n}{2}+m-1}{i} + \dbinom{\frac{n}{2}+m-2}{i-2} \]
Which is true, since $l \leq 2m-1$.
\end{proof}

4. Now we allow $l = 1$. This is proved in the paper easily. 

5. Now we assume that there is no matching.

We use the Lemma given in the paper: Let $G$ be a bipartite graph on $U \cup V$ with minimum degree $\delta_U \geq 1$ in $U$ and maximum degree $\Delta_V$ in $V$. Suppose there is no matching from $U$ to $V$. Then there exist nonempty subsets $U_1 \subset U$ and $V_1 \subset V$ with a perfect matching $M: U_1 \to V_1$ and $e(U_1,V) + E(U \setminus U_1,V_1) \leq |U_1|\Delta_V$.

We claim that the auxiliary graph satisfies the conditions of the lemma with $U = \A, V = \B, \Delta_V = |U^l(M(A)) \cap F|$.

We define $\A_1$ and $\B_1$ and $\tilde{\F}$ as in the paper, and we get that the number of chains we are losing is $|\A_1||\delta^l A| - e(\A_1,\B)$ (where $A$ can be any set of size $n/2 + m$). Same argument from above shows that $|\delta^l A| > |U^l(M(A)) \cap F|$, so the same argument from the paper can be copied. In particular, we conclude that $\tilde{\F}$ has fewer 2-chains than $\F$, contradicting the optimality of $\F$.

\end{document}